% 設定『目錄』名稱
\renewcommand{\contentsname}{目錄}
\renewcommand{\listfigurename}{圖目錄}
\renewcommand{\listtablename}{表目錄}
%\renewcommand{\appendixtocname}{附錄}   % if \appendix  is used.
\renewcommand{\appendixname}{附錄}   % if \appendix  is used.
\renewcommand{\tablename}{表}
\renewcommand{\figurename}{圖}
\renewcommand{\bibname}{參考文獻}

\hypersetup{
    bookmarks=true,         % show bookmarks bar?
    colorlinks=true,        % false: boxed links; true: colored links
    linkcolor=red,          % color of internal links (change box color with linkbordercolor)
    citecolor=red,          % color of links to bibliography
    filecolor=cyan,         % color of file links
    urlcolor=magenta        % color of external links
}

%\newcommand{\img}{C:/Dropbox/ntpu_thesis/plot/}%如果所有圖檔存放在其他地方,先定義位置

\pagestyle{fancy}
\fancyhf{}
\renewcommand{\chaptermark}[1]{\markboth{\thechapter .\ #1}{}}  % 去除章編號前後的字
\titleformat{\chapter}[display]{\center\LARGE\sf}{第\ \thechapter\ 章}{0.2cm}{}  %設計章節標題式樣,標題置中
\titlespacing{\chapter}{0pt}{-50pt}{25pt}   %設計章節標題式樣,控制間距
%\fancyhead[RO,RE]{\leftmark}   %章節標題於頁眉/頁足上
\fancyfoot[CO,CE]{\thepage}

\renewcommand{\headrulewidth}{0pt}  % 頁眉下方的橫線
%\renewcommand{\footrulewidth}{0pt} % 設定頁首多一條粗細是 0.4 pt 的水平線

% 設定itemize符號
\renewcommand{\labelitemi}{$\bullet$}
\renewcommand{\labelitemii}{$\circ$}

%設定英文字型,不設的話就會使用預設的字型
\setmainfont{Times New Roman}

% 設定中文字體
\setCJKmainfont{標楷體} %設定中文為系統上的字型,而英文不去更動,使用原TeX字型
\XeTeXlinebreaklocale "zh"
\XeTeXlinebreakskip = 0pt plus 1pt %這兩行一定要加,中文才能自動換行


\renewcommand{\baselinestretch}{1.25}   %依照文章預設行距增加為 1.25倍(不同字型大小行距值,加大為1.25倍)

% 以下是目錄章節後面打點格式的設定:http://vardesa.blog.hexun.com.tw/58537832_d.html
\makeatletter
\def\@bfdottedtocline#1#2#3#4#5{%
\ifnum #1>\c@tocdepth \else
\vskip \z@ \@plus.2\p@
{\leftskip #2\relax \rightskip \@tocrmarg \parfillskip -\rightskip
\parindent #2\relax\@afterindenttrue
\interlinepenalty\@M
\leavevmode \bfseries
\@tempdima #3\relax
\advance\leftskip \@tempdima \null\nobreak\hskip -\leftskip
{#4}\normalfont\nobreak
\leaders\hbox{$\m@th
\mkern \@dotsep mu\hbox{.}\mkern \@dotsep
mu$}\hfill
\nobreak
\hb@xt@\@pnumwidth{\hfil\normalfont \normalcolor #5}%
\par}%
\fi}
\renewcommand*\l@chapter{\@bfdottedtocline{0}{0em}{1.5em}}
\makeatother

% 定義『各章節標題、圖表、頁尾註記』字型

\theoremstyle{plain}   % 排版格式,{plain}:最醒目格式
\newtheorem{thm}{定理}  % 將 Theorem 改為國字「定理」
\newtheorem{thmm}{定義}


%
%   調整內縮長度(依據學校規定與不同版面、字型調整)
%
\parindent=0.85cm
